%%%%% Latex document containing the section Two Travellers for the paper ``Planning an Optimal Trip to Europe''.

%%%%% Date created:   27 May 2015
%%%%% Date modified   27 May 2015
%%%%% Created by:     John Gilbertson


\documentclass[12pt]{article}

\usepackage{latexsym,amssymb,amsmath,epsfig,amsfonts,graphicx,url,pdflscape,lipsum}

\setlength{\topmargin}{-10pt}
\setlength{\headsep}{0pt}
\setlength{\headheight}{0pt}
\setlength{\textheight}{680pt}
\setlength{\oddsidemargin}{0pt}
\setlength{\evensidemargin}{0pt}
\setlength{\textwidth}{460pt}
\setlength{\parskip}{.30cm}
\parskip=10pt

\title{Two Travellers}
\author{}
\date{}

\begin{document}

\maketitle

\section{Meeting Up with Another Traveller}

Often a person decides to travel to Europe because their friend has planned a trip. The aim of the second person's trip is to spend as many days with the first person as possible. However, due to budget constraints, it might not be possible for the second person to spend their whole trip with the first person.

This problem can be solved using our model. We introduce a new variable $T_{id}$, which is a binary variable that equals 1 if the two people are together in city $i$ on day $d$. We also add an additional subscript $p$ to the location and move variables to indicate the person they refer to [i.e., $x_{idp}$ and $m_{ftdp}$]. When the second person tries to maximise the number of days they are with the first person, their objective function is:

\begin{equation*}
	\max \sum_{i \in \mathcal{C}} \sum_{d = 1}^{D} T_{id}
\end{equation*}

For this model, we need the additional constraint:

\begin{align}
	\sum_{p = 1}^{2} x_{idp} & \geq 2 \cdot T_{id} & \forall i \in \mathcal{C}, d = 1, \ldots, D
\end{align}

As we are maximising $T_{id}$, this constraint will ensure that $T_{id} = 1$ if both $x_{id1} = 1$ and $x_{id2} = 1$ (i.e., both people are in city $i$ on day $d$), otherwise it is $T_{id} = 0$.

This problem requires five optimisation steps:

\begin{enumerate}
	\item Maximise utility of Person 1's trip
	\item Minimise cost of Person 1's trip given that utility
	\item Maximise number of days where Person 1 and Person 2 are together given Person 1's trip
	\item Maximise utility of Person 2's trip given that number of days together
	\item Minimise cost of Person 2's trip given that utility and number of days with Person 1
\end{enumerate}

The first two optimisation steps decide the first person's trip independently of the second person. The last three optimisation steps decide the second person's trip given that the first person's trip has already been decided.

To reduce computational time, this model was run over 10 days with constant utility on each day (i.e., the initial utility values of each city in the base model). The first person's budget is $\$5,000$, and the second person's budget is $\$3,500$. The results are presented in the two tables below.

\begin{table}[ht!]
	\centering
	\begin{minipage}{0.48\textwidth}
		\centering
		\begin{tabular}{| c | c || c |}
			\hline
			Days & Number & City \\ \hline \hline
			1-4 & 4 & London \\ \hline
			5-6 & 2 & Berlin \\ \hline
			7-10 & 4 & Paris \\ \hline \hline
			\multicolumn{2}{| c ||}{Total Cost ($\$$)} & 4,355 \\ \hline
			\multicolumn{2}{| c ||}{Total Utility} & 944.00 \\ \hline
		\end{tabular}
		\caption{Person 1}
		\label{person_1_meetup}
	\end{minipage}
	\hfill
	\begin{minipage}{0.48\textwidth}
		\centering
		\begin{tabular}{| c | c || c |}
			\hline
			Days & Number & City \\ \hline \hline
			1-2 & 2 & London \\ \hline
			3-6 & 4 & Berlin \\ \hline
			7-10 & 4 & Istanbul \\ \hline \hline
			\multicolumn{2}{| c ||}{Total Cost ($\$$)} & 3,417 \\ \hline
			\multicolumn{2}{| c ||}{Total Utility} & 856.73 \\ \hline
		\end{tabular}
		\caption{Person 2}
		\label{person_2_meetup}
	\end{minipage}
\end{table}

As the second person has a smaller budget, they are only able to spend 4 out of the 10 days with the first person. The two people are together on days 1-2 in London, and on days 5-6 in Berlin. The second person's trip is $\$938$ cheaper than the first person's trip, but has a $10\%$ lower utility.

\section{Avoiding Another Traveller}

The previous section explored the problem of the second person wanting to spend as many days with the first person as possible. A similar could be that the second person minimise the number of days that they are in the same city as the first person.

This problem can be solved using a similar model. The new objective function is:

\begin{equation*}
	\min \sum_{i \in \mathcal{C}} \sum_{d = 1}^{D} T_{id}
\end{equation*}

The additional constraint in the previous section needs to be adjusted to:

\begin{align}
	\sum_{p = 1}^{2} x_{idp} & \leq 1 + T_{id} & \forall i \in \mathcal{C}, d = 1, \ldots, D
\end{align}

As we are minimising $T_{id}$, this constraint will ensure that $T_{id} = 1$ if both $x_{id1} = 1$ and $x_{id2} = 1$ (i.e., both people are in city $i$ on day $d$), otherwise $T_{id} = 0$.

This problem requires the same five optimisation steps as the previous section, except step 3 is now a minimisation step.

\begin{enumerate}
	\item Maximise utility of Person 1's trip
	\item Minimise cost of Person 1's trip given that utility
	\item \textbf{Minimise} number of days where Person 1 and Person 2 are together given Person 1's trip
	\item Maximise utility of Person 2's trip given that number of days together
	\item Minimise cost of Person 2's trip given that utility and number of days with Person 1
\end{enumerate}

The model was run over 10 days with constant utility, and both people having a budget of $\$4,200$. The results are presented in the two tables below.

\begin{table}[ht!]
	\centering
	\begin{minipage}{0.48\textwidth}
		\centering
		\begin{tabular}{| c | c || c |}
			\hline
			Days & Number & City \\ \hline \hline
			1-4 & 4 & London \\ \hline
			5-8 & 4 & Berlin \\ \hline
			9-10 & 2 & Paris \\ \hline \hline
			\multicolumn{2}{| c ||}{Total Cost ($\$$)} & 4,169 \\ \hline
			\multicolumn{2}{| c ||}{Total Utility} & 933.7 \\ \hline
		\end{tabular}
		\caption{Person 1}
		\label{person_1_avoid}
	\end{minipage}
	\hfill
	\begin{minipage}{0.48\textwidth}
		\centering
		\begin{tabular}{| c | c || c |}
			\hline
			Days & Number & City \\ \hline \hline
			1-2 & 2 & Istanbul \\ \hline
			3-4 & 2 & Berlin \\ \hline
			5-8 & 4 & Paris \\ \hline
			9-10 & 2 & London \\ \hline \hline
			\multicolumn{2}{| c ||}{Total Cost ($\$$)} & 4,132 \\ \hline
			\multicolumn{2}{| c ||}{Total Utility} & 899.01 \\ \hline
		\end{tabular}
		\caption{Person 2}
		\label{person_2_avoid}
	\end{minipage}
\end{table}

The second person is able to afford a trip where they are never in the same city on the same day as the first person . However, the second person's trip has a $4\%$ lower utility than the first person's trip even though both trips cost a similar amount. The second person is foregoing $4\%$ of their potential utility in order to never spend a day in the same city as the first person.

\end{document}