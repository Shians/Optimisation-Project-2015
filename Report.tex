%%%%% Latex document containing the paper ``Planning an Optimal Trip to Europe''.

%%%%% Date created:   13 May 2015
%%%%% Date modified   15 May 2015
%%%%% Created by:     John Gilbertson, Shian Su, Ria Szeredi and Kenneth Young


\documentclass[12pt]{article}

\usepackage{latexsym,amssymb,amsmath,epsfig,amsfonts,graphicx,url,pdflscape,lipsum,multicol}
\usepackage{algorithmic, algorithm}

\setlength{\topmargin}{-10pt}
\setlength{\headsep}{0pt}
\setlength{\headheight}{0pt}
\setlength{\textheight}{680pt}
\setlength{\oddsidemargin}{0pt}
\setlength{\evensidemargin}{0pt}
\setlength{\textwidth}{460pt}
\setlength{\parskip}{.30cm}
\parskip=10pt

%%%%%%%%%%%%%%%%%%%%%%%%%%%%%%%%%%%%%%%%%%%%%%%%%%%%%%%%%%%%%%%%%%%%%%%%%%%%%
%%%%%%%%%%%%%%%%%%%%%%%%%%%%%%%%%%%%%%%%%%%%%%%%%%%%%%%%%%%%%%%%%%%%%%%%%%%%%
%%%%%%%%%%%%%%%%%%%%%%%%%%%%%%%%%%%%%%%%%%%%%%%%%%%%%%%%%%%%%%%%%%%%%%%%%%%%%

\begin{document}

\clearpage
\vspace*{3cm}
\begin{center}
  {\large \bf Planning an Optimal Trip to Europe}\\[+10pt]
  John Gilbertson, Shian Su, Ria Szeredi and Kenneth Young\\[+10pt]
%%%%%%%%%%%%%%%%%%%%%%%%%%%%%%%%%%%%%%%%%%%%%%%%%%%%%%%%%%%%%%%%%%%%%%%%%%%%
%%%%%%%%%%%%%%%%%%%%%%%%%%%%%%%%%%%%%%%%%%%%%%%%%%%%%%%%%%%%%%%%%%%%%%%%%%%%
%%%%%%%%%%%%%%%%%%%%%%%%%%%%%%%%%%%%%%%%%%%%%%%%%%%%%%%%%%%%%%%%%%%%%%%%%%%%
  {\large \bf Abstract}\\[+10pt]
  \parbox{13cm}{\lipsum[3]}
\\[+20pt]
\parbox{13cm}{
Keywords: personalised tourist guide, linear programming, transportation planning, trip generation, decision making}
\\[+20pt]
\end{center}
\vfill
\clearpage

\tableofcontents
\pagebreak

%%%%%%%%%%%%%%%%%%%%%%%%%%%%%%%%%%%%%%%%%%%%%%%%%%%%%%%%%%%%%%%%%%%%%%%%%%%%
%%%%%%%%%%%%%%%%%%%%%%%%%%%%%%%%%%%%%%%%%%%%%%%%%%%%%%%%%%%%%%%%%%%%%%%%%%%%
%%%%%%%%%%%%%%%%%%%%%%%%%%%%%%%%%%%%%%%%%%%%%%%%%%%%%%%%%%%%%%%%%%%%%%%%%%%%

\section{Introduction}
\label{sec:intro}

%%%%%%%%%%%%%%%%%%%%%%%%%%%%%%%%%%%%%%%%%%%%%%%%%%%%%%%%%%%%%%%%%%%%%%%%%%%%

\begin{quote} \textit{
Introduction to the business you have selected. Which activity of the business have you selected to
improve on? Why?
} \end{quote}

\lipsum[3]


\pagebreak
%%%%%%%%%%%%%%%%%%%%%%%%%%%%%%%%%%%%%%%%%%%%%%%%%%%%%%%%%%%%%%%%%%%%%%%%%%%%
%%%%%%%%%%%%%%%%%%%%%%%%%%%%%%%%%%%%%%%%%%%%%%%%%%%%%%%%%%%%%%%%%%%%%%%%%%%%
%%%%%%%%%%%%%%%%%%%%%%%%%%%%%%%%%%%%%%%%%%%%%%%%%%%%%%%%%%%%%%%%%%%%%%%%%%%%
\section{Data} 
\label{sec:data}

%%%%%%%%%%%%%%%%%%%%%%%%%%%%%%%%%%%%%%%%%%%%%%%%%%%%%%%%%%%%%%%%%%%%%%%%%%%%

For the sake of brevity, we shall only provide data sets for the case of five cities here. This will be enough to demonstrate how we stored and used our data. The full data set can be found in the appendicies. All values are in Australian Dollars.

Our model will need to know the flight costs between all the cities. Initially we assumed that flights were uniform across all days of the trip and all times of each day. Our flight data was sourced from \url{momondo.com.au} \cite{momondo}. We searched for only the cheapest flights between any two cities which imposed more assumptions on our problem.

\begin{table}[h]
\caption{Airfares between Five Cities}
\centering
\vspace{1mm}
\begin{tabular}{c|c|c|c|c|c}
\hline
\rule{0pt}{2ex}  & Moscow & Paris & London & Madrid & Rome \\
\hline
\rule{0pt}{2ex}Moscow & 0 & 146 & 126 & 202 & 146 \\
Paris & 227 & 0 & 60 & 143 & 93 \\
London & 213 & 82 & 0 & 249 & 160 \\
Madrid & 188 & 86 & 136 & 0 & 144 \\
Rome & 223 & 80 & 125 & 146 & 0 \\

\end{tabular}
\end{table}

For simplicity we assumed that you can fly directly between all cities, even though in some cases the cheapest flight was not direct. This assumption is vaild as we are not attempting to find the optimal route around Europe, but instead minimise the overall cost of the trip or maximise the enjoyment gained. 

Aside from the cost of airfares, the cost of living is another contributor to our model's objective value. All our daily living expenses data were sourced from \url{budgetyourtrip.com} \cite{budget}. To simplify the computation of our daily costs, we have assumed that there is no reduction in cost if a traveller books accommodation at a single hotel over multiple nights.

\begin{table}[h]
\caption{Daily Costs of Five Cities}
\centering
\vspace{1mm}
\begin{tabular}{c|c|c|c|c|c}
\hline
\rule{0pt}{2ex}  & Moscow & Paris & London & Madrid & Rome \\
\hline
\rule{0pt}{2ex}Low & 36 & 81 & 110 & 57 & 68 \\
Mid & 92 & 223 & 298 & 148 & 169 \\
High & 233 & 657 & 845 & 393 & 423 \\

\end{tabular}
\end{table}

The last contribution to our model's objective value came from the flights from and to Melbourne at the begining and end of the trip. We assumed that the traveller bought a return ticket, and so this will force the first and last city that they visit to be the same. This is a reasonable assumption to make as the cost of purchasing a return ticket was generally less than that of purchasing two separate one-way tickets.  This flight data was again sourced from \url{momondo.com.au} \cite{momondo}.

\begin{table}[h]
\caption{Return Flight Costs of Five Cities}
\centering
\vspace{1mm}
\begin{tabular}{c|c|c|c|c}
\hline
\rule{0pt}{2ex} Moscow & Paris & London & Madrid & Rome \\
\hline
\rule{0pt}{2ex} 2311 & 2155 & 1908 & 1870 & 2584 \\

\end{tabular}
\end{table}

We assumed that all flights occur in the morning. We make this simplification as if the flight was at midday then the daily cost would have contributions from the city the traveller was leaving and the city they are going to. Further to this, we assumed that the return flight to Melbourne occurred on the morning after the last day of the trip. This was again to simplify the calculation of daily costs. If the traveller leaves a city in the early morning then we assume that there is no living expenses for that day.

\begin{table}[h]
\caption{Base Utilities of Five Cities}
\centering
\vspace{1mm}
\begin{tabular}{c|c|c|c|c}
\hline
\rule{0pt}{2ex} Moscow & Paris & London & Madrid & Rome \\
\hline
\rule{0pt}{2ex} 71 & 93 & 100 & 77 & 85 \\
\end{tabular}
\end{table}
The true utility of staying in a city is highly subjective and very difficult to measure. So for a basic model our utility is based on the overall popularity of a city as measured by a "bednights" statistic, which measures the number of nights tourists stayed in a given city. Because the difference in the raw data was quite large, the fifth root was taken such that all values were of similar magnitude and then rescaled to be a rounded percentage of the highest utility city. This reflects the behaviour of a naive traveller who makes their decisions on general popularity, later we explore specific preferences.

\pagebreak
%%%%%%%%%%%%%%%%%%%%%%%%%%%%%%%%%%%%%%%%%%%%%%%%%%%%%%%%%%%%%%%%%%%%%%%%%%%%
%%%%%%%%%%%%%%%%%%%%%%%%%%%%%%%%%%%%%%%%%%%%%%%%%%%%%%%%%%%%%%%%%%%%%%%%%%%%
%%%%%%%%%%%%%%%%%%%%%%%%%%%%%%%%%%%%%%%%%%%%%%%%%%%%%%%%%%%%%%%%%%%%%%%%%%%%
\section{Solution Methodology} 
\label{sec:methods}

%%%%%%%%%%%%%%%%%%%%%%%%%%%%%%%%%%%%%%%%%%%%%%%%%%%%%%%%%%%%%%%%%%%%%%%%%%%%

\subsection{Heuristics}

This problem has many similarities with the Travelling Salesman Problem (TSP), which is known not to be computationally efficient to solve. To avoid this issue, it can be useful to solve the problem using a greedy heuristic.

\subsubsection{Cheap Heuristic}

The cheap heuristic attempts to find the cheapest 15 day trip with 15 possible cities. For each city, the heuristic calculates the cost of travelling from Melbourne to that city and remaining in that city for maxDays. The algorithm then chooses the city that has the minimum cost. In the same way, cities are added recursively until there are no days left. The traveller then flies back to Melbourne from their final city.

\begin{algorithm}[ht!]
\caption{Cheap Heuristic}
\begin{algorithmic}
\STATE Begin in Melbourne
\FOR {each city $i$}
\STATE $\text{cost}(i) = \text{costFromMelb}(i) + \text{maxDays} \times \text{costDaily}(i)$
\ENDFOR
\STATE Go to city $i$ with the minimum cost and stay for maxDays
\STATE $\text{daysLeft} = \text{days} - \text{maxDays}$
\WHILE {$\text{daysLeft} > 0$}
\STATE $step = \min (\text{maxDays}, \text{daysLeft})$
\FOR {each city $i$ not yet visited}
\STATE $\text{cost}(i) = \text{costTravel}(\text{currentCity}, i) + step \times \text{costDaily}(i)$
\ENDFOR
\STATE Go to city $i$ with the minimum cost and stay for $step$ days
\STATE Decrement daysLeft by $step$
\ENDWHILE
\STATE Return to Melbourne from final city
\end{algorithmic}
\end{algorithm}

Using maxDays $= 4$ and the medium daily costs defined in Section DATA, the output of this heuristic is shown in Table~\ref{cheap_heuristic_output}. The heuristic solution is to spend the first 4 days in Istanbul, then fly to Moscow for 4 days, followed by Prague for 4 days and finally Venice for 3 days. The cost of this trip is $\$4,060$ including Melbourne flights, flights within Europe and daily costs. The utility of this trip is $1,102$.

\begin{table}[ht!]
	\centering
	\begin{tabular}{| c || c |}
		\hline
		Number of Days & City \\ \hline \hline
		4 & Istanbul \\ \hline
		4 & Moscow \\ \hline
		4 & Prague \\ \hline
		3 & Venice \\ \hline
	\end{tabular}
	\caption{Cheap Heuristic Output}
	\label{cheap_heuristic_output}
\end{table}

\subsubsection{Maximum Utility Heuristic}

The maximum utility heuristic attempts to find the 15 day trip with the maximum utility regardless of cost. At each iteration, the algorithm chooses the city with the maximum utility from the set of cities containing the unvisited cities and the current city. After spending a day in any city, the utility of staying another day in that city is multiplied by a decay factor.

\begin{algorithm}[ht!]
\caption{Maximum Utility Heuristic}
\begin{algorithmic}
\STATE Begin in Melbourne
\FOR {$j = 1, \ldots,$ days}
\STATE Find city $i$ with the maximum utility in the set $\{ \text{unvisited cities} \} \bigcup \{ \text{current city} \}$
\STATE Go to city $i$
\STATE Reduce the utility of the current city by the decay factor
\ENDFOR
\STATE Return to Melbourne from final city
\end{algorithmic}
\end{algorithm}

Using a decay factor of 0.98 and the medium daily costs defined in Section DATA, the output of this heuristic is shown in Table~\ref{max_utility_heuristic_output}. The utility of this trip is $1,330$, which is $21\%$ greater than the utility of the trip found using the cheap heuristic. The cost of this trip is $\$5,450$, which is $34\%$ ($\$1,390$) greater than the cheap heuristic.

\begin{table}[ht!]
	\centering
	\begin{tabular}{| c || c |}
		\hline
		Number of Days & City \\ \hline \hline
		4 & London \\ \hline
		4 & Paris \\ \hline
		2 & Berlin \\ \hline
		4 & Rome \\ \hline
		1 & Barcelona \\ \hline
	\end{tabular}
	\caption{Maximum Utility Heuristic Output}
	\label{max_utility_heuristic_output}
\end{table}
 
\pagebreak
\subsection{Mixed-Integer Linear Program}

\subsubsection{Base model}

% melbourne flight data

\textbf{Variables} \\
$x_{id}$: [binary] equal to 1 if the traveller is in city $i$ on day $d$. \\ 	% location variable
$y_{i}$: [binary] equal to 1 if the traveller ever visits city $i$. \\		% city variable
$m_{ijd}$: [binary] equal to 1 if the traveller moves from city $i$ to city $j$ on day $d$. \\	% move variable

\textbf{Parameters}\\
$D$: the total number of days. \\
$\mathcal{C}$: the set of cities available for visiting. \\
$\mu_{ij}$: the cost to move from city $i$ to city $j$. \\
$\delta_{i}$: the daily cost of city $i$. \\
$\alpha$: the minimum number of days allowed in a city. \\
$\omega$: the maximum number of days allowed in a city. \\

\begin{equation*}
\text{Min } \sum_{d=1}^{D} \sum_{i \in \mathcal{C}} \delta_{i} \cdot x_{id} + \sum_{d=1}^{D-1} \sum_{i \in \mathcal{C}} \sum_{j \in \mathcal{C}} \mu_{ij} \cdot m_{ijd}\\
\end{equation*}
\begin{align}
\text{s.t.~~~~~~~~} \sum_{d=1}^{D} x_{id} & \geq \alpha \cdot y_{i} & \forall i \in \mathcal{C}\\
\sum_{d=1}^{D} x_{id} & \leq \omega \cdot y_{i} &  \forall i \in \mathcal{C}\\
x_{id} + x_{j(d+1)} & \leq 1 + m_{ijd} & \forall i,j \in \mathcal{C}, d = 1..D-1 \\
\sum_{i \in \mathcal{C}} x_{id} &= 1 & d = 1..D \\
\sum_{i \in \mathcal{C}} \sum_{j \in \mathcal{C}} m_{ijd} &= 1 & d = 1..D-1 \\
\sum_{d=1}^{D} \sum_{j \in \mathcal{C}} m_{ijd} &\leq 1 + \sum_{d=1}^{D} m_{iid} & \forall i \in \mathcal{C} \\
x_{j1} + \sum_{d=1}^{D} \sum_{i \in \mathcal{C}} m_{ijd} &\leq 1 + \sum_{d=1}^{D} m_{jjd} & \forall j \in \mathcal{C}\\
x_{id}, y_{i}, m_{ijd} & \in \{0,1\} & \forall i, j \in \mathcal{C}, d = 1..D
\end{align}

The base model is a simple minimisation of cost during a trip through europe. It's clear that the optimal solution would be to simply fly to the cheapest city and stay there for the duration of the planned trip. So we set some limits on how long we can stay in a particular city.

\newpage
\subsubsection{Decaying Enjoyment}
Assuming that the enjoyment of staying in a city decays with the number of days stayed. We add the following to the model.

\textbf{Variables}\\
$s_{ij}$: [binary] equal to 1 if traveller stays in city i for j days.\\

\textbf{Parameters} \\
$u_{ij}$: The total utility of staying in city i for j days.\\

Then we no longer need (2) since we hope that the traveller naturally leaves the city once the marginal enjoyment is low enough but we retain (1) since it's still reasonable to set a length of minimum stay.

Then the new model is 
\begin{equation*}
\text{Max } \sum_{i \in \mathcal{C}} \sum_{j=1}^{D} s_{ij} * u_{ij}\\
\end{equation*}
\begin{align}
\text{s.t.~~~~~~~~}
\sum_{d=1}^{D} x_{id} & \geq \alpha \cdot y_{i} & \forall i \in \mathcal{C}\\
\sum_{j=1}^{D} s_{ij} &= y_{i} & \forall i \in \mathcal{C}\\
s_{ij} \cdot j &\leq \sum_{d=1}^{D} x_{id} & \forall i \in \mathcal{C}, j=1...D\\
x_{id} + x_{j(d+1)} & \leq 1 + m_{ijd} & \forall i,j \in \mathcal{C}, d = 1..D-1 \\
\sum_{i \in \mathcal{C}} x_{id} &= 1 & d = 1..D \\
\sum_{i \in \mathcal{C}} \sum_{j \in \mathcal{C}} m_{ijd} &= 1 & d = 1..D-1 \\
\sum_{d=1}^{D} \sum_{j \in \mathcal{C}} m_{ijd} &\leq 1 + \sum_{d=1}^{D} m_{iid} & \forall i \in \mathcal{C} \\
x_{j1} + \sum_{d=1}^{D} \sum_{i \in \mathcal{C}} m_{ijd} &\leq 1 + \sum_{d=1}^{D} m_{jjd} & \forall j \in \mathcal{C}\\
x_{id}, y_{i}, m_{ijd} & \in \{0,1\} & \forall i, j \in \mathcal{C}, d = 1..D
\end{align}\\
With this we find the optimal enjoyment possible for the traveller irrespective of cost, then we optimise again to minimise cost at the optimal enjoyment.
\begin{equation*}
\text{Min } \sum_{d=1}^{D} \sum_{i \in \mathcal{C}} \delta_{i} \cdot x_{id} + \sum_{d=1}^{D-1} \sum_{i \in \mathcal{C}} \sum_{j \in \mathcal{C}} \mu_{ij} \cdot m_{ijd}\\
\end{equation*}
\begin{align*}
\text{s.t.~~~~~~~~} & \\
\text{(1)-(9) from previous} &\text{ model is statisfied and} \\
\sum_{i \in \mathcal{C}} \sum_{j=1}^{D} s_{ij} * u_{ij} &= (\sum_{i \in \mathcal{C}} \sum_{j=1}^{D} s_{ij} * u_{ij})_{optimal}
\end{align*}\\
We look at the new constraints introduced.\\
\vspace{5mm}\\
\begin{tabular}{c|p{11cm}}
\hline
Constraint &  Explanation \\
\hline
$\sum_{j=1}^{D} s_{ij} = y_{i}$ & This ensures that only one stay duration is valid for each city, also that the stay durations are zero if a city is not visited. \\
c & d \\
\hline
$s_{ij} \cdot j \leq \sum_{d=1}^{D} x_{id}$ & When trying to maximise utility the program will try to set j to the highest value it can for each i because $u_{ij}$ is set up as the cumulative utility of staying in city $i$ for $j$ days. As a result this sets $s_{ij}$ 1 if we stay in city $i$ for $j$ days. \\
\end{tabular}
\vspace{5mm}\\
% \begin{tabular}{c|c}
% Constraint &  Explanation \\
% $\sum_{j=1}^{D} s_{ij} = y_{i}$ & This ensures that only one stay duration is valid for each city, also that the stay durations are zero if a city is not visited.
% $s_{ij} \cdot j \leq \sum_{d=1}^{D} x_{id}$ & When trying to maximise utility the program will try to set j to the highest value it can for each i because $u_{ij}$ is set up as the cumulative utility of staying in city $i$ for $j$ days. As a result this sets $s_{ij}$ 1 if we stay in city $i$ for $j$ days.\\
% \end{tabular}\\

Now in practice, solving the model in Fico Xpress takes a substantial amount of time as the number of cities and days of the trip grows. So we impose some additional constraints, first we note that with unconstrained utility and costs, both the utility and costs are non-decreasing with the number of days, then we can solve a shorter trip and use its results as lower bounds for cost and utility in the longer trip. This improves the running time significantly for maximising utility, furthermore we note that for an optimal utility, the number of days in each city will not change in the optimal cost optimisation. Therefore we can use this as a constraint for improving the performance of the cost optimisation. Then we can add the constraint
\begin{align*}
	\sum\limits_{j=1}^D s_{ij} &= (\sum\limits_{j=1}^D s_{ij})_{optimal} & \forall i \in C.
\end{align*}
If we have multiple solutions that optimise utility then we can simply optimise cost for each of them and take the minimum.
\pagebreak

%%%%%%%%%%%%%%%%%%%%%%%%%%%%%%%%%%%%%%%%%%%%%%%%%%%%%%%%%%%%%%%%%%%%%%%%%%%%
%%%%%%%%%%%%%%%%%%%%%%%%%%%%%%%%%%%%%%%%%%%%%%%%%%%%%%%%%%%%%%%%%%%%%%%%%%%%
%%%%%%%%%%%%%%%%%%%%%%%%%%%%%%%%%%%%%%%%%%%%%%%%%%%%%%%%%%%%%%%%%%%%%%%%%%%%

\section{Extensions and Results Discussion} 
\label{sec:extensions}

%%%%%%%%%%%%%%%%%%%%%%%%%%%%%%%%%%%%%%%%%%%%%%%%%%%%%%%%%%%%%%%%%%%%%%%%%%%

Currently our base model takes limited input from the propective traveller; only the number of days of the trip and possibly the minimum and maximum number of days spent in any one city. To make the model more applicable in real world scenarios we propose some extensions which can handle more varied inputs.

As a brief example, let us consider a traveller who does not want to visit Istanbul. Adding the following set of constraints to the base model will impose the traveller's needs (Istanbul's index is 13),
$$	x_{13,d} = 0~~,~~ \forall d \in D	$$
The base model always returned Istanbul as one of the cities visited, as it had the overall cheapest flights. So the traveller's dislike of Istanbul has increased the minimal cost of their trip from \$3672 to \$3860.

\subsection{Varying Airfares}
So far, we have assumed that the cost of flights was uniform across all days. This extension's aim is to ensure our model returns more realistic results by considering how the cost of airfares vary across the week. We use \url{momondo.com.au} again to source the airfare data \cite{momondo}. To reduce computation times we arbitrarily assume that the trip begins on a Sunday.

%Initially, we aimed to only differentiate the flight costs between weekdays and weekends, however after inspecting the data, the cost variation did not appear so binary. Instead, 
To model the variation in airfares, the travel cost parameter $\mu_{ij}$ needs an additional dimension to store the day of the week. So we now redefine $\mu$ as follows,

$\mu_{ijd}$: The cost to fly from city $i$ to city $j$ on weekday $d$, $\forall i,j \in \mathcal{C}$, $d=1...7$.

If the size of $\mathcal{C}$ is $n$, then we now have a 3-dimensional array of size $n\times n\times 7$. To define an array in the Mosel language, one must specify each element and its corresponding index tuple. There are many ways of shortcutting this process, however to define the elements of a 3-dimensional array the shorthand becomes rather inelegant. In the case of 15 cities, the parameter definition section became longer than the model definition itself.

Seeking an alternate and more compact implementation, we decided to take the current flight data as the base price to travel between any two cities. Then construct a new parameter which stored how the flight cost was augmented for each day of the week. So now, for each of the cities, we have a size $n \times 7$ array which stores these augmentation values as percentages,

\textbf{Parameters} \\
$ai_{jd}$: The percentage increase in airfares from city $i$ to city $j$ on day $d$, $\forall i,j \in \mathcal{C}$, $d=1...7$.\\

To demonstrate this we provide Moscow's augmentation array, $a1_{jd}$. An entry of zero indicates that there is no increase in price for the flight on that day. The full table can be found in the appendicies.\\
\begin{table}[h]
	\caption{Airfare Augmentation Data for Moscow}
	\centering
	\vspace{1mm}
	\begin{tabular}{c|c|c|c|c|c|c|c}
		\hline
		\rule{0pt}{2ex} City & Sunday & Monday & Tuesday & Wednesday & Thursday & Friday & Saturday \\
		\hline
		\rule{0pt}{2ex}Moscow & 0 & 0 & 0 & 0 & 0 & 0 & 0 \\
		Paris & 4 & 1 & 0 & 3 & 8 & 13 & 10 \\
		London & 11 & 3 & 0 & 2 & 6 & 8 & 6 \\
		Madrid & 10 & 3 & 0 & 6 & 13 & 17 & 14 \\
		Rome & 4 & 1 & 0 & 3 & 11 & 12 & 12 \\
	\end{tabular}
\end{table}

Now adding this to the objective function gives
\begin{equation*}
\text{Min } \sum_{d=1}^{D} \sum_{i \in \mathcal{C}} \delta_{i} \cdot x_{id} + \sum_{d=1}^{D-1} \sum_{i \in \mathcal{C}} \sum_{j \in \mathcal{C}} m_{ijd} \cdot (\mu_{ij(d \text{ mod }7)} + 0.01 \cdot \mu_{ij(d \text{ mod }7)} \cdot ai_{jd}) \\
\end{equation*}

As an illustration, now consider a traveller planning a 12 day trip, who needs visit Florence and then travel to Vienna for the last 4 days of their trip. Adding the following constraints to the model,
\begin{align}
x_{10,8} &= 1 & \nonumber\\
x_{14,d} &= 1 & \forall d=9...12, \nonumber
\end{align}
we receive the results from the base model and varying airfares extension as listed below.

\begin{table}[ht]
	\begin{minipage}[b]{0.45\linewidth}
		\caption{Base Model Output}
		\centering
		\vspace{1mm}
		\begin{tabular}{| c || c |}
			\hline
			Number of Days & City \\ \hline \hline
			x & X \\ \hline
			x & X \\ \hline
			x & X \\ \hline
			4 & Vienna \\ \hline
		\end{tabular}
		\label{varying_airfares_example_1}
	\end{minipage}
	\hspace{0.5cm}
	\begin{minipage}[b]{0.45\linewidth}
		\caption{Varying Airfares Output}
		\centering
		\vspace{1mm}
		\begin{tabular}{| c || c |}
			\hline
			Number of Days & City \\ \hline \hline
			x & X \\ \hline
			x & X \\ \hline
			x & X \\ \hline
			4 & Vienna \\ \hline
		\end{tabular}
		\label{varying_airfares_example_2}
	\end{minipage}
\end{table}

The base model returns a total cost of \$XXX. Whereas with the variable airfare data added to the model, the minimum cost increases by \$ZZZ (\%XX.XX) to \$YYY. This cost increase comes from the \%35 increase in flights from Florence to Vienna on a Sunday and the \%XX increase in flights from Vienna to Melbourne on a Thursday.  So removing the assumption that flight costs are uniform across all days has given us a better estimate of the minimum cost.


\subsection{Activity Preference}


\subsection{Avoiding or Joining Another Traveller}


\subsection{Multiple Person Trip}




\pagebreak
%%%%%%%%%%%%%%%%%%%%%%%%%%%%%%%%%%%%%%%%%%%%%%%%%%%%%%%%%%%%%%%%%%%%%%%%%%%%
%%%%%%%%%%%%%%%%%%%%%%%%%%%%%%%%%%%%%%%%%%%%%%%%%%%%%%%%%%%%%%%%%%%%%%%%%%%%
%%%%%%%%%%%%%%%%%%%%%%%%%%%%%%%%%%%%%%%%%%%%%%%%%%%%%%%%%%%%%%%%%%%%%%%%%%%%

\section{Conclusion and Recommendations}
\label{sec:conc}

%%%%%%%%%%%%%%%%%%%%%%%%%%%%%%%%%%%%%%%%%%%%%%%%%%%%%%%%%%%%%%%%%%%%%%%%%%%

\begin{quote} \textit{
Write your conclusions here. Througout the whole document, do not forget to cite your sources \cite{example}.
} \end{quote}

\lipsum[3]

% recommend: query a database for flight data as needed - flight data is constantly changing because of exchange rates, competition etc. so storage of static information is not desired here.


\pagebreak
%%%%%%%%%%%%%%%%%%%%%%%%%%%%%%%%%%%%%%%%%%%%%%%%%%%%%%%%%%%%%%%%%%%%%%%%%%%%
%%%%%%%%%%%%%%%%%%%%%%%%%%%%%%%%%%%%%%%%%%%%%%%%%%%%%%%%%%%%%%%%%%%%%%%%%%%%
%%%%%%%%%%%%%%%%%%%%%%%%%%%%%%%%%%%%%%%%%%%%%%%%%%%%%%%%%%%%%%%%%%%%%%%%%%%%

\section{Acknowledgements}
\label{sec:acknow}

%%%%%%%%%%%%%%%%%%%%%%%%%%%%%%%%%%%%%%%%%%%%%%%%%%%%%%%%%%%%%%%%%%%%%%%%%%%

\begin{quote} \textit{
Acknowledge anyone (person, organisation etc.) who has contributed to your project.
} \end{quote}

alysson for being such a sexy beast



\pagebreak
%%%%%%%%%%%%%%%%%%%%%%%%%%%%%%%%%%%%%%%%%%%%%%%%%%%%%%%%%%%%%%%%%%%%%%%%%%%%
%%%%%%%%%%%%%%%%%%%%%%%%%%%%%%%%%%%%%%%%%%%%%%%%%%%%%%%%%%%%%%%%%%%%%%%%%%%%
%%%%%%%%%%%%%%%%%%%%%%%%%%%%%%%%%%%%%%%%%%%%%%%%%%%%%%%%%%%%%%%%%%%%%%%%%%%%


\section{Appendix}
\label{sec:appen}

Daily expenses table goes here

\pagebreak
 
\begin{landscape}
\begin{table}[h]
\caption{Airfares between Twenty Five Cities (Part I)}
\centering
\vspace{1mm}
\begin{tabular}{c|c|c|c|c|c|c|c|c|c|c|c}
\hline
\rule{0pt}{2ex} From$\backslash$ To & Moscow & Paris & London & Madrid & Rome & Crete & Barcelona & Berlin & Budapest & Florence & Amsterdam   \\
\hline
\rule{0pt}{2ex}Moscow & 0 & 146 & 126 & 202 & 146 & 168 & 143 & 156 & 207 & 291 & 154 \\
Paris & 227 & 0 & 60 & 143 & 93 & 123 & 139 & 110 & 162 & 85 & 52  \\
London & 213 & 82 & 0 & 249 & 160 & 284 & 135 & 107 & 163 & 224 & 124  \\
Madrid & 188 & 86 & 136 & 0 & 144 & 188 & 69 & 70 & 96 & 138 & 103  \\
Rome & 223 & 80 & 125 & 146 & 0 & 177 & 39 & 96 & 69 & 84 & 94  \\
Crete & 233 & 81 & 223 & 188 & 58 & 0 & 131 & 123 & 58 & 241 & 200 \\
Barcelona & 200 & 121 & 79 & 76 & 32 & 165 & 0 & 143 & 108 & 103 & 90 \\
Berlin & 130 & 53 & 81 & 97 & 83 & 153 & 110 & 0 & 83 & 214 & 92 \\
Budapest & 249 & 133 & 125 & 96 & 41 & 123 & 96 & 76 & 0 & 164 & 157 \\
Florence & 309 & 246 & 162 & 172 & 86 & 301 & 90 & 186 & 204 & 0 & 227 \\
Amsterdam & 155 & 158 & 73 & 141 & 136 & 211 & 117 & 106 & 125 & 229 & 0 \\
Prague & 112 & 122 & 100 & 112 & 48 & 193 & 116 & 108 & 138 & 144 & 97 \\
Istanbul & 123 & 101 & 127 & 154 & 67 & 173 & 122 & 86 & 41 & 235 & 66 \\
Vienna & 145 & 162 & 122 & 202 & 90 & 75 & 145 & 117 & 175 & 171 & 90 \\
Venice & 167 & 65 & 17 & 117 & 84 & 191 & 97 & 119 & 110 & 152 & 75 \\
Goreme & 172 & 206 & 205 & 237 & 137 & 253 & 181 & 142 & 86 & 522 & 136 \\
Lisbon & 282 & 80 & 77 & 50 & 135 & 220 & 107 & 123 & 117 & 173 & 55 \\
Nice & 278 & 127 & 119 & 143 & 48 & 220 & 50 & 90 & 138 & 173 & 121 \\
Reykjavik & 566 & 190 & 284 & 337 & 242 & 484 & 217 & 260 & 246 & 454 & 246 \\
Edinburgh & 233 & 194 & 71 & 135 & 204 & 371 & 179 & 115 & 151 & 390 & 168 \\
Dublin & 196 & 87 & 58 & 102 & 146 & 265 & 107 & 86 & 143 & 253 & 138 \\
Krakow & 191 & 135 & 94 & 126 & 142 & 286 & 107 & 176 & 138 & 181 & 143 \\
Copenhagen & 258 & 174 & 71 & 158 & 160 & 315 & 95 & 56 & 157 & 200 & 152 \\
Athens & 118 & 149 & 156 & 135 & 72 & 39 & 156 & 109 & 91 & 230 & 109 \\
Munich & 188 & 120 & 136 & 163 & 69 & 112 & 135 & 110 & 126 & 192 & 170 \\
\end{tabular}
\end{table}
\end{landscape}

\pagebreak

\begin{landscape}
\begin{table}[h]
\caption{Airfares between Twenty Five Cities (Part II)}
\centering
\vspace{1mm}
\begin{tabular}{c|c|c|c|c|c|c|c|c|c|c|c}
\hline
\rule{0pt}{2ex} From$\backslash$ To & Prague & Istanbul & Vienna & Venice & Goreme & Lisbon & Nice & Reykjavik & Edinburgh & Dublin & Krakow  \\
\hline
\rule{0pt}{2ex}Moscow & 146 & 115 & 131 & 129 & 167 & 283 & 200 & 406 & 266 & 263 & 236 \\
Paris & 165 & 150 & 140 & 68 & 375 & 178 & 114 & 240 & 124 & 106 & 152 \\
London & 171 & 197 & 179 & 152 & 267 & 205 & 160 & 183 & 116 & 116 & 177 \\
Madrid & 124 & 180 & 143 & 119 & 168 & 57 & 152 & 336 & 127 & 125 & 107 \\
Rome & 94 & 74 & 105 & 71 & 157 & 155 & 62 & 180 & 155 & 174 & 147 \\
Crete & 114 & 120 & 164 & 169 & 265 & 154 & 144 & 394 & 178 & 211 & 214 \\
Barcelona & 110 & 141 & 103 & 74 & 203 & 69 & 50 & 241 & 151 & 117 & 100 \\
Berlin & 123 & 99 & 176 & 104 & 197 & 118 & 108 & 245 & 100 & 102 & 128 \\
Budapest & 113 & 50 & 149 & 97 & 100 & 104 & 130 & 265 & 117 & 104 & 107 \\
Florence & 155 & 279 & 229 & 150 & 470 & 197 & 233 & 421 & 223 & 228 & 107 \\
Amsterdam & 85 & 121 & 155 & 95 & 200 & 137 & 73 & 252 & 88 & 101 & 134 \\
Prague & 0 & 85 & 152 & 65 & 186 & 174 & 135 & 256 & 134 & 114 & 141 \\
Istanbul & 108 & 0 & 59 & 65 & 33 & 140 & 126 & 337 & 143 & 162 & 149 \\
Vienna & 181 & 121 & 0 & 194 & 152 & 254 & 166 & 284 & 199 & 211 & 147 \\
Venice & 96 & 177 & 152 & 0 & 274 & 153 & 200 & 316 & 135 & 146 & 127 \\
Goreme & 206 & 27 & 55 & 230 & 0 & 257 & 130 & 384 & 189 & 198 & 440 \\
Lisbon & 178 & 260 & 161 & 123 & 441 & 0 & 66 & 328 & 142 & 108 & 174 \\
Nice & 96 & 180 & 157 & 128 & 224 & 116 & 0 & 313 & 175 & 97 & 189 \\
Reykjavik & 285 & 334 & 370 & 323 & 332 & 325 & 334 & 0 & 175 & 97 & 189 \\
Edinburgh & 140 & 294 & 370 & 127 & 613 & 174 & 168 & 192 & 0 & 39 & 139 \\
Dublin & 110 & 217 & 156 & 146 & 396 & 114 & 138 & 254 & 34 & 0 & 137 \\
Krakow & 180 & 247 & 197 & 205 & 362 & 171 & 139 & 341 & 133 & 119 & 0 \\
Copenhagen & 96 & 234 & 229 & 192 & 363 & 191 & 154 & 270 & 138 & 149 & 149 \\
Athens & 62 & 98 & 105 & 130 & 141 & 173 & 137 & 263 & 179 & 138 & 149 \\
Munich & 194 & 103 & 194 & 165 & 176 & 159 & 117 & 263 & 142 & 143 & 154 \\
\end{tabular}
\end{table}
\end{landscape}

\pagebreak

\begin{table}[h]
\caption{Airfares between Twenty Five Cities (Part III)}
\centering
\vspace{1mm}
\begin{tabular}{c|c|c|c}
\hline
\rule{0pt}{2ex} From$\backslash$ To & Copenhagen & Athens & Munich  \\
\hline
\rule{0pt}{2ex}Moscow &  162 & 141 & 152 \\
Paris & 75 & 226 & 68 \\
London & 130 & 145 & 133 \\
Madrid & 108 & 169 & 184 \\
Rome & 143 & 160 & 28 \\
Crete & 176 & 46 & 144 \\
Barcelona & 104 & 146 & 93 \\
Berlin & 43 & 174 & 110 \\
Budapest & 123 & 133 & 122 \\
Florence & 164 & 268 & 158 \\
Amsterdam & 60 & 128 & 183 \\
Prague & 86 & 143 & 153 \\
Istanbul & 51 & 70 & 83 \\
Vienna & 177 & 154 & 149 \\
Venice & 184 & 181 & 101 \\
Goreme & 111 & 120 & 147 \\
Lisbon & 192 & 171 & 124 \\
Nice & 103 & 175 & 87 \\
Reykjavik & 218 & 175 & 266 \\
Edinburgh & 159 & 273 & 220 \\
Dublin & 60 & 186 & 155 \\
Krakow & 110 & 176 & 220 \\
Copenhagen & 0 & 128 & 218 \\
Athens & 131 & 0 & 145 \\
Mucich & 131 & 166 & 0 \\
\end{tabular}
\end{table}
\pagebreak

For brevities sake, we shall only include the tables detailing the airfare augmentation data for Moscow and Florence.

\begin{table}[h]
\caption{Airfare Augmentation Data for Moscow}
\centering
\vspace{1mm}
\begin{tabular}{c|c|c|c|c|c|c|c}
\hline
\rule{0pt}{2ex} City & Sunday & Monday & Tuesday & Wednesday & Thursday & Friday & Saturday \\
\hline
\rule{0pt}{2ex}Moscow & 0 & 0 & 0 & 0 & 0 & 0 & 0 \\
Paris & 4 & 1 & 0 & 3 & 8 & 13 & 10 \\
London & 11 & 3 & 0 & 2 & 6 & 8 & 6 \\
Madrid & 10 & 3 & 0 & 6 & 13 & 17 & 14 \\
Rome & 4 & 1 & 0 & 3 & 11 & 12 & 12 \\
Crete & 2 & 0 & 0 & 2 & 7 & 13 & 5 \\
Barcelona & 12 & 3 & 0 & 7 & 13 & 16 & 14 \\
Berlin & 1 & 0 & 2 & 7 & 12 & 7 & 5 \\
Budapest & 0 & 2 & 2 & 7 & 10 & 7 & 5 \\
Florence & 5 & 0 & 5 & 15 & 19 & 15 & 7 \\
Amsterdam & 0 & 0 & 4 & 11 & 15 & 8 & 3 \\
Prague & 0 & 0 & 3 & 7 & 11 & 7 & 3 \\
Istanbul & 2 & 2 & 9 & 11 & 12 & 7 & 0 \\
Vienna & 0 & 0 & 7 & 12 & 16 & 14 & 7 \\
Venice & 2 & 0 & 5 & 14 & 22 & 14 & 6 \\
Goreme & 4 & 1 & 0 & 2 & 5 & 6 & 15 \\
Lisbon & 4 & 0 & 3 & 8 & 13 & 10 & 5 \\
Nice & 3 & 0 & 4 & 12 & 16 & 17 & 6 \\
Reykjavik & 1 & 0 & 6 & 4 & 6 & 7 & 1 \\
Edinburgh & 1 & 0 & 2 & 8 & 16 & 12 & 10 \\
Dublin & 6 & 0 & 3 & 10 & 20 & 17 & 15 \\
Krakow & 0 & 0 & 6 & 6 & 16 & 6 & 2 \\
Copenhagen & 0 & 0 & 3 & 6 & 10 & 10 & 10 \\
Athens & 2 & 0 & 3 & 6 & 13 & 15 & 3 \\
Mucich & 1 & 0 & 5 & 11 & 15 & 18 & 7 \\
\end{tabular}
\end{table}

\pagebreak
\begin{table}[h]
\caption{Airfare Augmentation Data for Florence}
\centering
\vspace{1mm}
\begin{tabular}{c|c|c|c|c|c|c|c}
\hline
\rule{0pt}{2ex} City & Sunday & Monday & Tuesday & Wednesday & Thursday & Friday & Saturday \\
\hline
\rule{0pt}{2ex}Moscow & 20 & 7 & 0 & 5 & 18 & 28 & 39 \\
Paris & 20 & 6 & 0 & 5 & 18 & 25 & 39 \\
London & 17 & 0 & 0 & 10 & 12 & 13 & 40 \\
Madrid & 17 & 10 & 3 & 0 & 4 & 5 & 9 \\
Rome & 0 & 15 & 2 & 11 & 8 & 2 & 8 \\
Crete & 5 & 0 & 2 & 3 & 4 & 15 & 6 \\
Barcelona & 6 & 0 & 2 & 7 & 10 & 9 & 11 \\
Berlin & 18 & 7 & 0 & 13 & 10 & 27 & 28 \\
Budapest & 1 & 2 & 2 & 4 & 2 & 0 & 5 \\
Florence & 0 & 0 & 0 & 0 & 0 & 0 & 0 \\
Amsterdam & 4 & 3 & 2 & 5 & 0 & 12 & 18 \\
Prague & 5 & 0 & 4 & 7 & 3 & 4 & 5 \\
Istanbul & 4 & 0 & 1 & 3 & 4 & 7 & 8 \\
Vienna & 8 & 2 & 0 & 5 & 12 & 15 & 35 \\
Venice & 5 & 0 & 5 & 15 & 23 & 14 & 14 \\
Goreme & 5 & 2 & 0 & 1 & 4 & 6 & 10 \\
Lisbon & 12 & 0 & 1 & 8 & 12 & 11 & 13 \\
Nice & 5 & 3 & 2 & 0 & 1 & 1 & 9 \\
Reykjavik & 24 & 12 & 0 & 1 & 5 & 6 & 26 \\
Edinburgh & 10 & 0 & 3 & 8 & 15 & 16 & 25 \\
Dublin & 5 & 1 & 0 & 2 & 2 & 5 & 11 \\
Krakow & 26 & 9 & 3 & 0 & 0 & 5 & 11 \\
Copenhagen & 10 & 0 & 1 & 4 & 7 & 15 & 22 \\
Athens & 8 & 0 & 0 & 3 & 7 & 16 & 13 \\
Mucich & 6 & 5 & 5 & 0 & 4 & 6 & 20 \\

\end{tabular}
\end{table}



\pagebreak
%%%%%%%%%%%%%%%%%%%%%%%%%%%%%%%%%%%%%%%%%%%%%%%%%%%%%%%%%%%%%%%%%%%%%%%%%%%%
%%%%%%%%%%%%%%%%%%%%%%%%%%%%%%%%%%%%%%%%%%%%%%%%%%%%%%%%%%%%%%%%%%%%%%%%%%%%
%%%%%%%%%%%%%%%%%%%%%%%%%%%%%%%%%%%%%%%%%%%%%%%%%%%%%%%%%%%%%%%%%%%%%%%%%%%%

\begin{thebibliography}{99}
%%%%%%%%%%%%%%
%%%%%%%%%%%%%%
\bibitem{momondo}
momondo.com.au,
\emph{Global Travel Search Engine}, accessed 6 May 2015.
Available online: \url{http://www.momondo.com.au/}
%%%%%%%%%%%%%%
%%%%%%%%%%%%%%
\bibitem{budget}
Budget Your Trip,
\emph{Trvavel Cost Search}, accessed 13 May 2015.
Available online: \url{http://www.budgetyourtrip.com/}
%%%%%%%%%%%%%%
%%%%%%%%%%%%%%
\bibitem{example}
Foo, Bar, Smith (2002)
\emph{Example Title}.
Example Publisher. ISBN XXX-XXX
%%%%%%%%%%%%%%
%%%%%%%%%%%%%%
\end{thebibliography}




%%%%%%%%%%%%%%%%%%%%%%%%%%%%%%%%%%%%%%%%%%%%%%%%%%%%%%%%%%%%%%%%%%%%%%%%%%%%
%%%%%%%%%%%%%%%%%%%%%%%%%%%%%%%%%%%%%%%%%%%%%%%%%%%%%%%%%%%%%%%%%%%%%%%%%%%%
%%%%%%%%%%%%%%%%%%%%%%%%%%%%%%%%%%%%%%%%%%%%%%%%%%%%%%%%%%%%%%%%%%%%%%%%%%%%
\end{document}